\documentclass[11pt]{article}
\usepackage[a4paper, margin=1in]{geometry}
\usepackage{parskip}
\usepackage{hyperref}
\usepackage{graphicx}
\usepackage{amsmath}

\title{\textbf{Reliable Multi-Agent Systems for Automated Technical Data Acquisition and Validation}}
\author{
  Yange Zheng \\
  Technical University of Munich \\
  Supervisor: [Supervisor Name] (TUM) \\
  Industry Supervisor: [Your Boss's Name] (BMW AG) \\
  Thesis Advisor: [PhD Candidate's Name] (BMW AG \& TUM)
}
\date{Start Date: January 6, 2025 \\
Submission Deadline: July 6, 2025}

\begin{document}
\maketitle

\tableofcontents
\newpage

\section{Introduction}

\subsection{Background and Industry Context}
Global automotive manufacturers such as BMW rely on timely, accurate technical data when evaluating electronic components for new vehicle platforms.  
A key artefact in this workflow is the \emph{Quotation Analysis Form}~(QAF), a semi-structured document submitted by suppliers that mixes free-text part descriptions, tentative prices, package options, and—often inconsistently—Manufacturer Part Numbers~(MPNs).  
Because QAF layouts vary by supplier and project, automated ingestion has proved difficult; engineers still copy–paste identifiers, search the web for datasheets, and manually transcribe critical specifications into internal databases.  
With thousands of parts assessed each month, the cumulative cost in engineer hours, delayed sourcing decisions, and downstream data errors is substantial.

\subsection{Problem Statement}
Current VBA and Python scripts at BMW handle isolated steps (e.\,g.\ basic web scraping) but fail when a field is missing, a datasheet URL changes, or the MPN is embedded inside noisy text.  
Moreover, there is no unified mechanism to:  
\begin{enumerate}
  \item robustly extract a clean MPN from a free-text QAF line,
  \item select the best data source among distributor APIs and web search results,
  \item parse heterogeneous PDF datasheets into machine-readable form, and
  \item verify that every extracted value is plausible and provenance-linked.
\end{enumerate}
The absence of such an end-to-end solution limits the scalability of data-driven cost estimation and component qualification.

\subsection{Research Objectives and Questions}
The overarching objective of this thesis is to design, implement, and evaluate a \emph{reliable multi-agent system} that converts noisy QAF inputs into validated, structured component specifications.  
Concretely, the work addresses three research questions:

\begin{description}
  \item[RQ1:] How accurately can a language-model agent extract MPNs from semi-structured QAF text?
  \item[RQ2:] How can multiple retrieval agents (distributor API, web search) be orchestrated to maximise datasheet coverage and minimise latency?
  \item[RQ3:] What is the field-level precision and recall of a vision-language parsing agent, and how do validation rules affect overall data quality?
\end{description}

\subsection{Contributions}
The thesis makes four practical contributions:
\begin{enumerate}
  \item A modular, fault-tolerant multi-agent architecture that coordinates language and vision models with external tools.
  \item An implementation that integrates GPT-4–based MPN extraction, Octopart API queries, Google search scraping, vision-language PDF parsing, and JSON-schema validation.
  \item A benchmark on 1\,000 real BMW QAF entries comprising metrics for MPN accuracy, datasheet retrieval success, and specification precision/recall.
  \item A downstream regression model demonstrating that the extracted specifications improve component price prediction—highlighting direct business value.
\end{enumerate}

\subsection{Thesis Structure}
The remainder of this document is organised as follows.  
Section~\ref{sec:relatedwork} reviews language-model agents, vision-language models, and prior document-AI systems in procurement.  
Section~\ref{sec:problem} formalises the data challenges inherent in QAF processing.  
Section~\ref{sec:architecture} details the multi-agent design, while Section~\ref{sec:implementation} discusses implementation specifics such as prompt engineering and API integration.  
Section~\ref{sec:evaluation} presents the experimental setup and results, and Section~\ref{sec:costmodel} evaluates the downstream cost-prediction model.  
Section~\ref{sec:discussion} analyses limitations and industrial implications.  
Finally, Section~\ref{sec:conclusion} concludes the thesis and outlines future research directions.


\section{Related Work}
\begin{itemize}
  \item Language model agents and tool use (ReAct, LangChain)
  \item Vision-language models for document understanding
  \item Multi-agent architectures in applied AI
  \item Document AI in procurement and engineering contexts
\end{itemize}

\section{Problem Setting}
\begin{itemize}
  \item Characteristics of QAFs (Quotation Analysis Forms)
  \item Challenges in MPN extraction and data heterogeneity
  \item Requirements for robustness, reliability, and validation
\end{itemize}

\section{System Architecture}
\begin{itemize}
  \item Overview of the multi-agent framework
  \item Agent roles: MPN extractor, retrieval (API + web), parser, validator
  \item Coordination strategy and tool integration (function-calling, fallback)
  \item Agent communication and memory structure
\end{itemize}

\section{Implementation}
\begin{itemize}
  \item LLM prompt design for MPN extraction
  \item Integration of Octopart API and Google-based datasheet search
  \item VLM (GPT-4 Vision) for PDF parsing and structured spec output
  \item Schema validation and provenance tracking
  \item Logging, metrics collection, and error handling
\end{itemize}

\section{Evaluation}
\begin{itemize}
  \item Dataset: 1,000 real QAFs, annotated spec samples (50--100)
  \item Metrics: MPN accuracy, spec precision/recall, latency, task success
  \item Evaluation setup and agent benchmarking
  \item Failure analysis and qualitative examples
\end{itemize}

\section{Downstream Application: Cost Modeling}
\begin{itemize}
  \item Feature engineering from extracted specs
  \item Regression model (e.g., XGBoost or linear) for price prediction
  \item Evaluation metrics (R\textsuperscript{2}, MAE)
  \item Interpretation and practical implications
\end{itemize}

\section{Discussion}
\begin{itemize}
  \item System performance summary
  \item Strengths and limitations of the current architecture
  \item Potential for integration into BMW workflows
  \item Reflection on data quality and generalizability
\end{itemize}

\section{Conclusion and Future Work}
\begin{itemize}
  \item Summary of contributions
  \item Key results and insights
  \item Future extensions: human-in-the-loop validation, feedback learning
\end{itemize}

\appendix
\section*{Appendices}
\begin{itemize}
  \item Example QAF inputs and outputs
  \item Annotation schema for specs
  \item Prompt templates and API call examples
  \item Full experimental results and error logs
\end{itemize}

\end{document}
