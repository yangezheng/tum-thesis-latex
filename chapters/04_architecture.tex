\chapter{System Architecture}
\label{chapter:architecture}

% Usage Instructions
% Insert this code in place of your current \section{System Architecture} placeholder.
% 
% Add \usepackage{tikz} and \usepackage{enumitem} if not already present.
% 
% Replace or augment the heuristic equation, table, or diagram details to match your final implementation.



This chapter describes the overall design of the proposed multi-agent pipeline, the responsibilities of each agent, their communication strategy, and the supporting infrastructure for storage, logging, and fault-recovery.

\section{High-Level Overview}

Figure~\ref{fig:architecture} illustrates the data flow from a raw QAF line to a validated specification record and a downstream cost prediction.  
Each rectangular node represents an autonomous agent implemented as a function-calling wrapper around a large language model (LLM) or vision-language model (VLM).  
Solid arrows indicate the primary execution path, while dashed arrows represent feedback loops used for retries and validation.


\begin{figure}[H]
  \centering
  \begin{tikzpicture}[
    node distance=1.7cm and 2.5cm,
    io/.style={rectangle, rounded corners=2pt, draw, align=center, minimum width=2.8cm, minimum height=1cm, font=\small},
    agent/.style={rectangle, draw, rounded corners=3pt, fill=blue!6, minimum width=3.7cm, minimum height=1.1cm, font=\small},
    datastore/.style={cylinder, draw, fill=gray!10, minimum height=1.1cm, minimum width=2cm, font=\small},
    every node/.style={font=\small}
  ]
    % Nodes
    \node[io] (qaf) {\shortstack{QAF entry \\ (raw text)}};
    \node[agent, right=of qaf] (mpn) {\shortstack{MPN\\Extraction\\Agent}};
    \node[agent, right=of mpn] (api) {\shortstack{Retrieval\\Agent\\(Octopart API)}};
    \node[agent, below=2.5cm of api] (web) {\shortstack{Retrieval\\Agent\\(Web search)}};
    \node[agent, right=of api] (parser) {\shortstack{Parsing\\Agent\\(VLM)}};
    \node[agent, right=of parser] (validator) {\shortstack{Validation\\Module}};
    \node[datastore, right=of validator] (db) {\shortstack{Spec DB}};
    \node[agent, below=2.5cm of validator] (cost) {\shortstack{Cost\\Model}};

    % Arrows
    \draw[->] (qaf) -- (mpn);
    \draw[->] (mpn) -- node[above]{MPN} (api);
    \draw[->] (mpn) |- (web);
    \draw[->] (api) -- (parser);
    \draw[->] (web) -| (parser);
    \draw[->] (parser) -- (validator);
    \draw[->] (validator) -- (db);
    \draw[->, dashed] 
      (validator.west) 
      -- ++(-1.0,0) 
      |- node[above, pos=0.25]{retry / fallback} (web.south);
    \draw[->] (validator) -- (cost);
  \end{tikzpicture}
  \caption{End-to-end architecture of the multi-agent pipeline.}
  \label{fig:architecture}
\end{figure}



\section{Agent Responsibilities}
\paragraph{MPN Extraction Agent}  
Receives a raw QAF string and applies a few-shot GPT-4 prompt that asks the model to return a JSON object with the most likely manufacturer part number.  
A lightweight post-processor removes package suffixes (e.\,g.\ “‐TR”) and checks against a regex whitelist of vendor prefixes.

\paragraph{Retrieval Agent (API)}  
Queries the Octopart REST API with the extracted MPN.  
If the API returns \verb|datasheets| with a direct PDF link and the file size is within 50 kB–20 MB, the link is forwarded to the Parsing Agent.

\paragraph{Retrieval Agent (Web)}  
Uses Playwright to perform a Google search of the form  
\texttt{\{MPN\} datasheet filetype:pdf}.  
% Results are scored by the reliability heuristic in Equation~\eqref{eq:heuristic}.  
The top-3 PDFs are downloaded for validation.

\paragraph{Parsing Agent}  
Given a PDF path, calls GPT-4 Vision with a chain-of-thought template: first summarise the datasheet sections, then extract \verb|max_voltage|, \verb|max_current|, \verb|power_rating|, \verb|temp_range|, and return JSON plus page/bbox evidence.

\paragraph{Validation Module}  
Executes three layers of checks:
\begin{enumerate}
  \item JSON-Schema validation (types + units),
  \item range rules (e.\,g.\ $0 < V_{max} < 1000V$),
  \item duplicate consistency (table vs. paragraph vs. Octopart value).
\end{enumerate}  
Failures trigger a dashed feedback edge to the Web Retrieval Agent for retry.

\paragraph{Cost Model}  
Trains an XGBoost regressor on validated specs + historical SAP prices to demonstrate the economic relevance of the structured output.

\section{Agent Coordination and Memory}
Agents communicate via the LangChain executor, which offers:  
\begin{itemize}
  \item \textbf{Function-calling interface}: each agent exposes a JSON schema; GPT-4 picks the correct function based on the task.  
  \item \textbf{Shared Memory}: RedisJSON stores intermediate results (\verb|mpn|, \verb|pdf_path|, validation flags) accessible to all agents.  
  \item \textbf{Retry Logic}: a decorator intercepts exceptions, waits exponentially (1 s, 2 s, 4 s), and reroutes control to the fallback agent if attempts $>3$.
\end{itemize}

\section{Source Ranking Heuristic}
For every candidate PDF URL $u$, the cost score is computed as
\begin{equation}
\label{eq:heuristic}
\mathrm{cost}(u)=0.2\;\mathrm{domain}(u)+0.4\;\mathrm{pdfWeight}(u)+0.4\;\mathrm{failRate}(u),
\end{equation}
where \(\mathrm{domain}=0\) for manufacturer sites, 1 for trusted distributors, 2 otherwise;  
\(\mathrm{pdfWeight}=0\) if the URL ends with \texttt{.pdf}, else 1;  
and \(\mathrm{failRate}\) is an exponentially decayed average of previous download failures from the same host.

\section{Data Storage and Logging}
Validated JSON records are written to a PostgreSQL table with columns for \verb|mpn|, \verb|spec_json| (JSONB), \verb|provenance| (JSONB), and an Azure Blob link to the PDF.  
OpenTelemetry traces (agent name, latency, token usage) are exported to Grafana for live monitoring.

\section{Fault Tolerance}
If both retrieval agents fail, the system stores a \verb|status = "unresolved"| entry for manual triage.  
Parsing errors with low OCR confidence ($<0.7$) raise a warning but still save partially validated fields, ensuring \emph{at-least-once} data capture.

\section{Security and Cost Considerations}
All external calls pass through BMW’s outbound proxy; API secrets are managed via Azure Key Vault.  
Estimated operating cost on an Azure D8s~v5 VM plus one A10 GPU is \(\approx 0.90\,€\) per 1 000 processed QAF lines, dominated by GPT-4 Vision tokens.

\section{Mapping to Requirements}
Table~\ref{tab:reqmap} summarises how architectural components satisfy the automation requirements defined in Chapter~\ref{sec:problem}.

\begin{table}[ht]
  \centering
  \caption{Requirements coverage by architectural component.}
  \label{tab:reqmap}
  \begin{tabular}{ll}
    \hline
    Requirement & Fulfilled by \\\hline
    R1 Robust MPN extraction & MPN Extraction Agent (LLM + regex) \\
    R2 Adaptive retrieval    & API + Web Retrieval Agents, heuristic ranking \\
    R3 Multi-modal parsing   & VLM Parsing Agent (GPT-4 Vision) \\
    R4 Schema validation     & Validation Module (JSON Schema, range rules) \\
    R5 Traceability          & Provenance storage (page, bbox) \\
    R6 Scalability           & Redis memory, async Playwright, GPU offload \\\hline
  \end{tabular}
\end{table}

Together, these design choices create a fault-tolerant, interpretable, and extensible architecture capable of meeting BMW’s data-quality and throughput requirements.
