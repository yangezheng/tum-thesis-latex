\chapter{System Architecture}
\label{chapter:architecture}

% Usage Instructions
% Insert this code in place of your current \section{System Architecture} placeholder.
% 
% Add \usepackage{tikz} and \usepackage{enumitem} if not already present.
% 
% Replace or augment the heuristic equation, table, or diagram details to match your final implementation.



This chapter describes the overall design of the proposed multi-agent pipeline, the responsibilities of each agent, their communication strategy, and the supporting infrastructure for storage, logging, and fault-recovery.

\section{High-Level Overview}

Figure~\ref{fig:architecture} illustrates the data flow from a raw standardised quotation analysis template line to a validated specification record and a downstream cost prediction.  
Each rectangular node represents an autonomous agent implemented as a function-calling wrapper around a large language model (LLM) or vision-language model (VLM).  
Solid arrows indicate the primary execution path, while dashed arrows represent feedback loops used for retries and validation.


\begin{figure}
  \centering
  \begin{tikzpicture}[
    node distance=1.5cm and 2.5cm,
    every node/.style={font=\small},
    start/.style={rectangle, rounded corners=3pt, draw, minimum width=4cm, minimum height=1cm, align=center},
    planner/.style={rectangle, rounded corners=3pt, draw, dashed, minimum width=6cm, minimum height=1.2cm, align=center},
    agent/.style={rectangle, draw, minimum width=5cm, minimum height=1.1cm, align=center, anchor=north},
    fusion/.style={rectangle, rounded corners=3pt, draw, dashed, minimum width=6cm, minimum height=1.2cm, align=center},
    validation/.style={rectangle, draw, minimum width=5cm, minimum height=1.1cm, align=center},
    storage/.style={cylinder, draw, minimum height=1.1cm, minimum width=2.5cm, align=center},
    cost/.style={rectangle, draw, minimum width=5cm, minimum height=1.1cm, align=center}
  ]
    % Top node
    \node[start] (start) {Start: MPN Input};

    % Planner
    \node[planner, below=of start] (planner) {Planner Agent -- MPN Extraction};

    % Two agents
    \node[agent, below left=1.5cm and 3.5cm of planner.south] (api) {Retrieval Agent -- API Access};
    \node[agent, below=1.5cm of planner] (pdf) {Retrieval Agent -- PDF Parsing};

    % Field Fusion
    \node[fusion, below=2.2cm of pdf] (fusion) {Field Fusion -- Schema Resolution};

    % Validation Module
    \node[validation, below=1.5cm of fusion] (validation) {Validation Module};

    % Storage
    \node[storage, below=1.5cm of validation] (storage) {Spec DB};

    % Cost Model
    \node[cost, below=1.5cm of storage] (cost) {Cost Model};

    % Arrows
    \draw[->] (start) -- (planner);
    \draw[->] (planner) -- (api);
    \draw[->] (planner) -- (pdf);
    \draw[->] (api) -- (fusion);
    \draw[->] (pdf) -- (fusion);
    \draw[->] (fusion) -- (validation);
    \draw[->] (validation) -- node[right]{valid} (storage);
    \draw[->] (storage) -- (cost);

    % Feedback loop for invalid case
    \draw[->, dashed]
      (validation.east)
      .. controls +(3,0) and +(3,0) .. (planner.north east)
      node[midway, right, align=center] {invalid};

  \end{tikzpicture}
  \caption{Multi-agent pipeline: vertical, layered architecture with validation and feedback.}
  \label{fig:architecture}
\end{figure}



\section{Agent Responsibilities}
\paragraph{Planner Agent}  
Receives a raw QAF string and applies a few-shot GPT-4 prompt that asks the model to return a JSON object with the most likely manufacturer part number.
The output is then passed to the Retrieval Agents.


\paragraph{Retrieval Agent (API)}  
Queries the Octopart REST API with the extracted MPN.  
If the API returns \verb|datasheets| with a direct PDF link and the file size is within 50 kB–20 MB, the link is forwarded to the Parsing Agent.

\paragraph{Retrieval Agent (PDF)}  
Uses Playwright to perform a Google search of the form  
\texttt{\{MPN\} datasheet filetype:pdf}.  
% Results are scored by the reliability heuristic in Equation~\eqref{eq:heuristic}.  
The most likely PDFs are downloaded for validation.
after the pdfs are downloaded, the pdf agent will use a chain of thought prompt to extract the most likely parameters from the pdf.

\paragraph{Field Fusion}
The field fusion module will take the output from the retrieval agents and compare them to the standardised quotation analysis template.
The field will output a json object with the most likely parameters, the output of the field fusion module is then passed to the validation module.

\paragraph{Validation Module}  
Executes three layers of checks:
\begin{enumerate}
  \item JSON-Schema validation (types + units),
  \item range rules (e.\,g.\ $0 < V_{max} < 1000V$),
\end{enumerate}  
Failures trigger a dashed feedback edge to the Web Retrieval Agent for retry.

\paragraph{Cost Model Example}  
As a sample downstream task, a cost model is implemented by training an random forest regressor on validated specification records and historical prices. This demonstrates how the structured output can be used for economic analysis or price prediction.

\section{Agent Coordination and Memory}
Agents are orchestrated using the LangGraph framework, which enables flexible, graph-based workflows and agent composition.  
\begin{itemize}
  \item \textbf{Function-calling interface}: Each agent exposes a JSON schema; GPT-4 (or other LLM) selects and invokes the appropriate function based on the current task node in the graph.
  \item \textbf{Shared State}: LangGraph maintains a mutable state object that is passed between agents as the workflow progresses. This state contains intermediate results (such as \verb|mpn|, \verb|pdf_path|, validation flags) and is accessible to all agents at each step, enabling coordination without explicit external storage.
  \item \textbf{Retry Logic}: Retry and fallback behavior is implemented within the graph structure—nodes can be configured to handle exceptions, apply exponential backoff (1 s, 2 s, 4 s), and reroute control to fallback agents if the number of attempts exceeds 3.
\end{itemize}

\section{Source Ranking Heuristic}
For every candidate PDF URL $u$, the cost score is computed as
\begin{equation}
\label{eq:heuristic}
\mathrm{cost}(u)=0.2\;\mathrm{domain}(u)+0.4\;\mathrm{pdfWeight}(u)+0.4\;\mathrm{failRate}(u),
\end{equation}
where \(\mathrm{domain}=0\) for manufacturer sites, 1 for trusted distributors, 2 otherwise;  
\(\mathrm{pdfWeight}=0\) if the URL ends with \texttt{.pdf}, else 1;  
and \(\mathrm{failRate}\) is an exponentially decayed average of previous download failures from the same host.

\section{Data Storage and Logging}
Validated JSON records are written to a PostgreSQL table with columns for \verb|mpn|, \verb|spec_json| (JSONB), \verb|provenance| (JSONB), and an Azure Blob link to the PDF.  
OpenTelemetry traces (agent name, latency, token usage) are exported to Grafana for live monitoring.

\section{Fault Tolerance}
If both retrieval agents fail, the system stores a \verb|status = "unresolved"| entry for manual triage.  
Parsing errors with low OCR confidence ($<0.7$) raise a warning but still save partially validated fields, ensuring \emph{at-least-once} data capture.

\section{Security and Cost Considerations}
All external calls pass through BMW's outbound proxy; API secrets are managed via Azure Key Vault.  
Estimated operating cost on an Azure D8s~v5 VM plus one A10 GPU is \(\approx 0.90\,€\) per 1 000 processed QAF lines, dominated by GPT-4 Vision tokens.

\section{Mapping to Requirements}
Table~\ref{tab:reqmap} summarises how architectural components satisfy the automation requirements defined in Chapter~\ref{sec:problem}.

\begin{table}[ht]
  \centering
  \caption{Requirements coverage by architectural component.}
  \label{tab:reqmap}
  \begin{tabular}{ll}
    \hline
    Requirement & Fulfilled by \\\hline
    R1 Robust MPN extraction & MPN Extraction Agent (LLM + regex) \\
    R2 Adaptive retrieval    & API + Web Retrieval Agents, heuristic ranking \\
    R3 Multi-modal parsing   & VLM Parsing Agent (GPT-4 Vision) \\
    R4 Schema validation     & Validation Module (JSON Schema, range rules) \\
    R5 Traceability          & Provenance storage (page, bbox) \\
    R6 Scalability           & Redis memory, async Playwright, GPU offload \\\hline
  \end{tabular}
\end{table}

Together, these design choices create a fault-tolerant, interpretable, and extensible architecture capable of meeting BMW's data-quality and throughput requirements.
