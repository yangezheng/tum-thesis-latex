\chapter{Discussion}
\label{chapter:discussion}

%--------------------------------------------------
\section{Key Findings}
The preceding chapters demonstrate that a multi-agent pipeline can transform noisy quotation-template entries into high-quality specification data and deliver tangible business value via downstream cost modelling.  Three insights stand out:
\begin{enumerate}
  \item \textbf{Tool use amplifies LLM accuracy}. Function-calling agents equipped with retrieval and vision tools close the gap between open-ended reasoning and structured data extraction.
  \item \textbf{Validation layers are essential}. JSON-Schema, range checks, and cross-source consistency filters reduced field-level error by 5–8~percentage points without noticeable latency overhead.
  \item \textbf{Structured specs have predictive power}. Chapter~\ref{chapter:costmodel} shows that technical parameters alone explain over 70\,\% of price variance for electronic components.
\end{enumerate}

%--------------------------------------------------
\section{Limitations}
\subsection{Data Coverage}
Although the corpus covers thousands of part numbers, it is biased toward passive and digital ICs.  Electro-mechanical parts and high-power modules remain under-represented.

\subsection{Model Generalisation}
GPT-4 Vision handled most datasheets, yet failed on scans with heavy watermarking or handwritten annotations.  Future versions may need OCR fine-tuning or synthetic augmentation.

\subsection{Cost and Privacy}
Running a vision model at scale incurs non-trivial token costs and raises document-privacy concerns, especially when datasheets include confidential errata.

%--------------------------------------------------
\section{Industrial Implications}
\begin{itemize}
  \item \textbf{Engineer Throughput}: Automating the extraction of technical specifications from quotation templates can reduce manual triage and data entry time by approximately 60–70\,\%, based on our evaluation. This allows engineering teams to focus on design, validation, and supplier communication rather than repetitive documentation tasks.
  \item \textbf{Supplier Negotiations}: By generating early, data-driven cost estimates directly from extracted specifications, procurement teams can establish more accurate price anchors during Request for Quotation (RFQ) rounds. This reduces reliance on supplier-provided data and supports more objective, evidence-based negotiations.
  \item \textbf{Quality Assurance}: The pipeline attaches provenance metadata—such as source URLs and datasheet page numbers—to each extracted parameter. This enables traceable audit trails for quality management and compliance, without imposing additional manual effort on staff.
\end{itemize}

%--------------------------------------------------
\section{Future Work}
\begin{enumerate}
  \item \textbf{Marking Code Recognition}: Develop an upstream vision-based scanner to extract marking codes directly from component images, then map these codes to Manufacturer Part Numbers (MPNs) before entering the main pipeline. This would enable processing of loose or unlabelled parts, broadening applicability beyond cases where the MPN is explicitly available. Integrating a robust marking code–to–MPN mapping database and handling ambiguous or partial codes will be key challenges.
  \item \textbf{Active Learning}: Incorporate human-in-the-loop feedback to iteratively improve the system. This could involve retraining prompts, updating ranking heuristics, or fine-tuning models on low-confidence or misclassified cases. Building an interface for efficient expert review and correction will accelerate adaptation to new component types and edge cases.
  \item \textbf{Edge Deployment and Model Diversity}: Explore deployment of quantised vision-language models (VLMs) for on-premises inference, addressing data-sovereignty and privacy requirements where cloud APIs are not viable. In addition to GPT-4 Vision, evaluate alternative models such as open-source VLMs (e.g., LLaVA, MiniGPT-4, or BLIP-2) for datasheet and marking code recognition, balancing accuracy, latency, and cost for industrial use.
  \item \textbf{Dynamic Pricing Models}: Integrate commodity indices and supply-chain risk signals into the cost model for real-time updates, enabling more responsive and market-aware price predictions.
\end{enumerate}

%--------------------------------------------------
\section{Conclusion}
The discussion highlights both the strengths and shortcomings of the proposed system.  While current results meet accuracy and latency targets, scaling to broader component classes and tightening privacy guarantees remain open challenges.  Addressing these will be crucial for transitioning from a promising prototype to a production-grade procurement assistant.