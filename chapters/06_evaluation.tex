\chapter{Evaluation}
\label{chapter:evaluation}

%--------------------------------------------------
\section{Experimental Setup}
All experiments were conducted on a MacBook Pro 16-inch with an Apple M2 Pro processor and 16~GB RAM. The quotation–template corpus consists of 800 unique lines drawn from four recent procurement projects, covering microcontrollers, power ICs, and passive components.

\subsection{Ground-Truth Annotation}
Each template line was manually annotated to establish high-quality ground truth. The annotation process consisted of:
\begin{enumerate}
  \item Identifying the correct Manufacturer Part Number (\textbf{MPN}),
  \item Compiling a list of 26 technical parameters for each part,
  \item Recording the values of five key electronic parameters: voltage, current, power, number of pins, and package.
\end{enumerate}
Annotations were performed by carefully reading datasheets and entering the required information into a structured spreadsheet.

\subsection{Baselines}
To fairly assess the performance of our end-to-end pipeline, we compare against several strong baselines, each representing a classical or ablated approach:
\begin{itemize}
  \item \textbf{RegexOnly}: Applies handcrafted regular expressions to extract the MPN from text, isolating the challenge of part number identification without any model-based reasoning.
  \item \textbf{GPT-3.5 (various zero/multi-shot settings)}: Includes zero-shot ($T=0.0$), zero-shot ($T=0.7$), and multi-shot ($T=0.7$) variants of GPT-3.5 for MPN extraction. This group of baselines demonstrates the effect of prompt engineering and decoding strategy on LLM-based extraction, and provides a fair comparison to our method.
  \item \textbf{APIOnly}: Directly queries the Octopart API using the extracted MPN, without attempting web search or PDF fallback if the API response is incomplete. This simulates a typical single-source retrieval workflow, lacking robustness to missing or partial data.
  \item \textbf{PyPDF2}: Uses the PyPDF2 Python library to extract all text from datasheet PDFs, then applies rule-based heuristics (regular expressions and keyword matching) to identify and extract technical parameters. No language model or agent-based reasoning is used, simulating a fully classical, non-LLM pipeline for end-to-end extraction.
\end{itemize}

%--------------------------------------------------
\section{Evaluation Metrics}
\begin{description}
  \item[\textbf{MPN~Accuracy}] Fraction of entries for which the extracted Manufacturer Part Number exactly matches the ground truth. (Compared between RegexOnly, GPT-3.5 variants, and Ours.)
  \item[\textbf{Field Coverage Rate}] Average number of non-null fields per row, divided by the total possible fields (26). (Compared between APIOnly and Ours.)
  \item[\textbf{Field~F1}] For the five most important technical parameters, the proportion extracted correctly; a parameter is considered correct if its value matches the ground truth within a $\pm2\,\%$ tolerance after unit normalisation. (Compared between PyPDF2 and Ours.)
\end{description}

%--------------------------------------------------
\section{Results}
\subsection{MPN Extraction (RQ1)}
\begin{table}[H]
\centering
\caption{MPN extraction performance (RegexOnly, GPT-3.5 variants).}
\label{tab:mpn}
\begin{tabular}{lc}
\toprule
Method & MPN~Accuracy \\
\midrule
RegexOnly                        & 0.63 \\
GPT-3.5~(Zero-Shot, $T=0.7$)     & 0.85 \\
GPT-3.5~(Zero-Shot, $T=0.0$)     & 0.87 \\
GPT-3.5~(Multi-Shot, $T=0.7$)    & 0.96 \\
Ours (GPT-3.5, Multi-Shot, $T=0.0$)       & \textbf{0.97} \\
\bottomrule
\end{tabular}
\end{table}
Our agent-based pipeline achieves substantially higher MPN accuracy than RegexOnly. RegexOnly often fails on non-standard formats, while our method is robust to such variations.

\subsection{Field Coverage Rate (RQ2)}
\begin{table}[H]
\centering
\caption{Field coverage rate (APIOnly vs. Ours).}
\label{tab:coverage}
\begin{tabular}{lc}
\toprule
Method & Field Coverage Rate \\
\midrule
APIOnly   & 0.27 \\
Ours      & \textbf{0.79} \\
\bottomrule
\end{tabular}
\end{table}
Our pipeline achieves much higher field coverage than APIOnly, which is limited by incomplete distributor data and does not attempt fallback strategies.

\subsection{Parameter Extraction (RQ3)}
\begin{table}[H]
\centering
\caption{Field-level F1 score for key parameters (PyPDF2 vs. Ours).}
\label{tab:fields}
\begin{tabular}{lccccc}
\toprule
Method & Voltage & Current & Power & Pins & Package \\
\midrule
PyPDF2    & 0.49 & 0.45 & 0.36 & 0.51 & 0.54 \\
Ours      & \textbf{0.93} & \textbf{0.94} & \textbf{0.91} & \textbf{0.95} & \textbf{0.96} \\
\bottomrule
\end{tabular}
\end{table}
Our pipeline delivers near-human accuracy on all five key parameters, with the largest gains observed for power and package fields. The PyPDF2 baseline struggles with ambiguous layouts and inconsistent units.

\section{Threats to Validity}
\begin{itemize}
  \item \textbf{Data Bias}: While datasheets for electronic components are often easier to parse due to their semi-structured format, this may inflate performance metrics and limit generalisability to domains such as mechanical parts, which typically have more variable and unstructured documentation.
  \item \textbf{Model Drift}: The GPT-4o model used for PDF parsing may change over time; to mitigate this, outputs were cached at evaluation time.
  \item \textbf{Annotation Noise}: All ground-truth annotations were performed by a single expert, so some undetected errors may remain.
\end{itemize}

%--------------------------------------------------
\section{Summary}
The proposed pipeline outperforms regex, API-only, and PyPDF2 baselines in extraction accuracy and robustness, achieving near-human performance on parameter extraction. These results support the subsequent cost-prediction analysis in Chapter~\ref{chapter:costmodel}.