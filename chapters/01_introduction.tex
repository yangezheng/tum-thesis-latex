\chapter{Introduction}\label{chapter:introduction}

%--------------------------------------------------
\begin{figure}[H]
  \centering
  \fbox{\begin{minipage}{0.95\textwidth}
  \textbf{Example quotation template line (synthetic)}\\[0.5em]
  \texttt{
  10UF 25V X5R 0603 C1608X5R1E106K080AC RoHS Reel 4000pcs \$0.073
  }
  \end{minipage}}
  \caption{Typical supplier-submitted quotation template line (synthetic example).}
  \label{fig:standardised-quotation-analysis-template-line-example}
\end{figure}

\section{Background }
The digitalisation of automotive supply chains has accelerated in recent years, yet many data-critical workflows still depend on semi-structured documents and manual interpretation.  
A common example in automotive supply chains is a \emph{standardised quotation analysis template}, a file exchanged during supplier negotiations that lists part designations, preliminary prices, delivery conditions, and essential technical details.  
Although such templates frequently include an identifier resembling a Manufacturer Part Number~(MPN), suppliers often embed the MPN within longer descriptive strings—alongside electrical characteristics and ordering information—or omit it altogether.  
As shown in Figure~\ref{fig:standardised-quotation-analysis-template-line-example}, lines in these templates typically appear as unstructured, single-line entries that conflate multiple attributes in free-form text.  
Consequently, engineers must manually inspect each line, identify or correct the MPN, locate the corresponding datasheet, and extract key specifications—such as maximum voltage or operating temperature—for input into internal cost and qualification systems.  
With several thousand parts assessed per program phase, this \emph{ad hoc} process consumes hundreds of engineer-hours and remains susceptible to transcription errors.

%--------------------------------------------------
\section{Problem Statement }
Previous attempts to automate the processing of quotation analysis templates primarily relied on regular expressions to extract Manufacturer Part Numbers~(MPNs) from raw supplier text.  
While effective in narrowly defined cases, these approaches were brittle---minor variations in formatting, inconsistent delimiters, or embedded product descriptions frequently caused extraction to fail.  
A further limitation was ambiguity: when the regular expression returned a null result, it was unclear whether no MPN was present or if the pattern simply failed to match a valid one.  
To address these issues, large language models~(LLMs) offer a more robust alternative.  
By reasoning over free-form text and recognizing contextual cues, LLMs can identify MPNs even when they are embedded in descriptive strings or surrounded by supplier-specific notations.

%--------------------------------------------------
\section{Research Objectives }
The central objective of this thesis is to design, implement, and evaluate a \textbf{reliable multi-agent system} that transforms noisy inputs from quotation analysis templates into validated, structured component specifications.  
To operationalise this goal, the work addresses three research questions:

\begin{description}
\item[\textbf{RQ1.}] How accurately can a language-model agent extract MPNs from noisy, multilingual part designation fields in semi-structured quotation analysis templates?

\item[\textbf{RQ2.}] How reliably can web-search agents retrieve the correct PDF datasheet for a given part, especially when distributor APIs are incomplete or unavailable?

\item[\textbf{RQ3.}] How accurately can technical parameters be extracted from retrieved datasheets using vision-language models or other automated methods?
\end{description}

%--------------------------------------------------
\section{Proposed Approach }
To answer these questions, the thesis proposes a modular pipeline whose agents each address a single sub-task:  
\begin{enumerate}
  \item An \textbf{MPN Extraction Agent} uses GPT-4 with prompt engineering and regex to isolate MPN from quotation template text.  
  \item Two \textbf{Retrieval Agents} operate in parallel:
    \begin{itemize}
      \item[(i)] An \emph{API Agent} queries the Octopart distributor API to obtain structured part metadata and datasheet links.
      \item[(ii)] A \emph{Datasheet Agent} performs a Google-based search to locate datasheet PDFs directly, prioritising high-confidence domains (e.g., manufacturer or distributor sites). And leverages Vision Language Models (VLMs) to extract the key electrical specifications from the retrieved datasheets.
    \end{itemize}
  \item A \textbf{Validation Module} enforces JSON-Schema constraints and physical range rules and flags low-confidence extractions for manual review.  
  \item Finally, a \textbf{Cost-Prediction Model} is trained on the extracted specifications, demonstrating the economic value of the pipeline's output.
\end{enumerate}

%--------------------------------------------------
\section{Contribution }
The thesis offers four concrete contributions:
\begin{enumerate}
  \item A fault-tolerant multi-agent framework that integrates LLMs, web search, distributor APIs, and VLM parsing into a single orchestrated workflow.  
  \item Empirical benchmarks reporting MPN extraction accuracy, datasheet retrieval success, field-level precision/recall, and end-to-end latency.  
  \item A downstream cost-prediction experiment showing that structured specs extracted by the system improve price-estimation accuracy, thus underlining direct business impact.
\end{enumerate}

%--------------------------------------------------
\section{Thesis Structure }
The remainder of this document is organised as follows:  
Chapter~\ref{chapter:relatedwork} surveys literature on language-model agents, vision-language models, and document AI in procurement.  
Chapter~\ref{chapter:problem} formalises the challenges inherent to quotation-template processing.  
Chapter~\ref{chapter:architecture} details the proposed multi-agent architecture, while Chapter~\ref{chapter:implementation} describes implementation specifics, including prompts, API wrappers, and validation logic.  
Chapter~\ref{chapter:evaluation} presents the experimental setup and quantitative results.  
Section~\ref{chapter:costmodel} demonstrates the downstream cost-modelling application.  
Chapter~\ref{chapter:discussion} reflects on limitations and industrial implications, and Chapter~\ref{chapter:conclusion} concludes the thesis with avenues for future work.
