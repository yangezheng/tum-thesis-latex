\chapter{Introduction}\label{chapter:introduction}

\begin{figure}[H]
  \centering
  \fbox{\begin{minipage}{0.95\textwidth}
  \textbf{Example QAF Line (synthetic)}\\[0.5em]
  \texttt{
  10UF 25V X5R 0603 C1608X5R1E106K080AC RoHS Reel 4000pcs \$0.073
  }
  \end{minipage}}
  \caption{Typical supplier-submitted QAF line (synthetic example).}
  \label{fig:qaf-line-example}
\end{figure}

\section{Background \ding{44}}
The digitalisation of automotive supply chains has accelerated in recent years, yet many data-critical workflows still depend on semi-structured documents and manual interpretation.  
A prominent example at BMW is the \emph{Quotation Analysis Form}~(QAF), a file exchanged during supplier negotiations that lists part designations, preliminary prices, delivery conditions, and essential technical details.  
Although QAFs frequently include an identifier resembling a Manufacturer Part Number~(MPN), suppliers often embed the MPN within longer descriptive strings—alongside electrical characteristics and ordering information—or omit it altogether.  
As shown in Figure~\ref{fig:qaf-line-example}, QAF lines typically appear as unstructured, single-line entries that conflate multiple attributes in free-form text.  
Consequently, engineers must manually inspect each line, identify or correct the MPN, locate the corresponding datasheet, and extract key specifications—such as maximum voltage or operating temperature—for input into BMW’s internal cost and qualification systems.  
With several thousand parts assessed per program phase, this \emph{ad hoc} process consumes hundreds of engineer-hours and remains susceptible to transcription errors.

\section{Problem Statement}
Preliminary attempts to automate QAF processing relied on regular expressions or supplier-specific parsing templates.  
Such approaches break when the supplier changes formatting, a datasheet link becomes unavailable, or unexpected language (e.\,g.\ German abbreviations) appears in the field.  
Modern language models (LLMs) and vision-language models (VLMs) promise the ability to reason over diverse text and visual layouts, yet their raw outputs can be inconsistent and lack provenance.  
A scalable solution must therefore \emph{coordinate multiple specialised tools}, recover gracefully from partial failures, and guarantee that extracted numeric values are plausible and traceable to the source document.

\section{Research Objectives}
The central objective of this thesis is to design, implement, and evaluate a \textbf{reliable multi-agent system} that transforms noisy QAF inputs into validated, structured component specifications.  
To operationalise this goal, the work addresses three research questions:

\begin{description}
  \item[RQ1.] How accurately can a language-model agent extract correct MPNs from semi-structured QAF strings?
  \item[RQ2.] Can cooperative retrieval agents—one using a distributor API, the other web search—achieve near-complete datasheet coverage while maintaining low latency?
  \item[RQ3.] What precision and recall can a vision-language parsing agent attain on key electrical specifications, and how do schema-based validation rules affect overall data quality?
\end{description}

\section{Proposed Approach}
To answer these questions, the thesis proposes a modular pipeline whose agents each address a single sub-task:  
\begin{enumerate}
  \item An \textbf{MPN Extraction Agent} uses GPT-4 with prompt engineering and regex post-filters to isolate standardised identifiers from QAF text.  
  \item Two \textbf{Retrieval Agents} query complementary sources: (i) the Octopart API for structured part metadata and (ii) a Google-backed web crawler prioritised by a cost-based heuristic that favours manufacturer domains.  
  \item A \textbf{Parsing Agent} leverages GPT-4~Vision to read PDF datasheets and return JSON fields—voltage, current, power rating, and temperature range—together with bounding-box provenance.  
  \item A \textbf{Validation Module} enforces JSON-Schema constraints and physical range rules and flags low-confidence extractions for manual review.  
  \item Finally, a \textbf{Cost-Prediction Model} (gradient-boosted regression) is trained on the extracted specifications, demonstrating the economic value of the pipeline’s output.
\end{enumerate}

\section{Contribution}
The thesis offers four concrete contributions:
\begin{enumerate}
  \item A fault-tolerant multi-agent framework that integrates LLMs, web search, distributor APIs, and VLM parsing into a single orchestrated workflow.  
  \item An implementation evaluated on \textbf{1\,000} real BMW QAFs, including a manually annotated subset of 50–100 datasheets for ground-truth specification labels.  
  \item Empirical benchmarks reporting MPN extraction accuracy, datasheet retrieval success, field-level precision/recall, and end-to-end latency.  
  \item A downstream cost-prediction experiment showing that structured specs extracted by the system improve price-estimation accuracy, thus underlining direct business impact.
\end{enumerate}

\section{Thesis Structure}
The remainder of this document is organised as follows:  
Section~\ref{chapter:relatedwork} surveys literature on language-model agents, vision-language models, and document AI in procurement.  
Section~\ref{chapter:problem} formalises the challenges inherent to QAF processing.  
Section~\ref{chapter:architecture} details the proposed multi-agent architecture, while Section~\ref{sec:implementation} describes implementation specifics, including prompts, API wrappers, and validation logic.  
Section~\ref{chapter:evaluation} presents the experimental setup and quantitative results.  
Section~\ref{chapter:costmodel} demonstrates the downstream cost-modelling application.  
Section~\ref{chapter:discussion} reflects on limitations and industrial implications, and Section~\ref{sec:conclusion} concludes the thesis with avenues for future work.
