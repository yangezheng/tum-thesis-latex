\section{Related Work}
\label{chapter:relatedwork}

This chapter surveys prior research relevant to the proposed multi-agent pipeline, grouped into four areas: (i) tool-using language-model agents, (ii) vision-language models for document understanding, (iii) multi-agent orchestration frameworks, and (iv) document-AI applications in procurement and engineering.

%--------------------------------------------------
\subsection{Tool-Using Language-Model Agents}
Large language models have been extended with \emph{tool use} capabilities to overcome context-length limits and provide up-to-date information.  
ReAct~\cite{yao2023react} interleaves chain-of-thought reasoning with executable actions, while the OpenAI function-calling interface popularised structured tool invocation in production systems.  
LangChain\footnote{\url{https://www.langchain.com}} offers an open-source abstraction layer for wrapping external tools (APIs, SQL, web search) behind LLM calls.  
AutoGPT and Voyager~\cite{wang2023voyager} push autonomy further but lack guarantees on reliability—highlighting the need for validation mechanisms such as those proposed in this thesis.

%--------------------------------------------------
\subsection{Vision-Language Models for Document Understanding}
Document AI has progressed from OCR-centric pipelines to transformer-based models that combine layout and text.  
LayoutLMv3~\cite{xuhuang2022layoutlmv3}, DocPrompting~\cite{li2023docprompt}, and StrucTexT~\cite{li2021structext} incorporate visual and spatial cues for table extraction.  
Recent work extends general VLMs (BLIP-2, GPT-4 Vision) to technical documents, but evaluation on complex engineering datasheets remains limited.  
Our parsing agent adopts GPT-4 Vision and focuses on reliability and schema-validated output rather than raw captioning.

%--------------------------------------------------
\subsection{Multi-Agent Orchestration Frameworks}
Early agent systems in NLP used finite-state controllers; recent research revisits \emph{LLM-driven} agents.  
MetaGPT~\cite{hong2023metagpt} assigns specialised roles (product manager, engineer) to different LLM instances.  
CrewAI\footnote{\url{https://github.com/joaomdmoura/crewAI}} and OpenAI’s “Assistants” API provide abstractions for role-based coordination.  
Most studies focus on qualitative demos; quantitative benchmarks on noisy industrial tasks, such as QAF processing, are scarce.  
Our work contributes empirical results on orchestration efficacy (success, latency) in a production-grade setting.

%--------------------------------------------------
\subsection{Document-AI in Procurement and Engineering}
Prior industry case studies address invoice OCR, purchase-order matching, and part catalogue normalisation~\cite{chung2020invoiceai, koch2022bom}.  
Datasheet processing is typically rule-based or limited to text segments~\cite{rahman2021datasheet}.  
To our knowledge, no published system handles the complete \emph{QAF → datasheet → validated spec} loop using modern LLM / VLM techniques.  
This thesis fills that gap and releases a benchmark on 1 000 QAFs to foster reproducible research.

%--------------------------------------------------
\subsection{Research Gap}
\begin{itemize}
  \item Existing LLM agents excel in controlled benchmarks but lack validation and provenance when applied to safety-critical engineering data.
  \item VLMs demonstrate impressive captioning accuracy yet are rarely evaluated on fine-grained numeric specs inside technical PDFs.
  \item No prior work jointly optimises extraction \emph{and} downstream business value (e.g.\ cost prediction) within an end-to-end industrial pipeline.
\end{itemize}
These gaps motivate the multi-agent system and evaluation strategy proposed in the following chapters.
