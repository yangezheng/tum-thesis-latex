\chapter{Problem Setting}
\label{chapter:problem}

\section{Characteristics of Material Pricing Sheets}
A Material Pricing Sheet is a semi-structured document exchanged during early supplier negotiations.  
In many companies, such templates appear in spreadsheet format.  
Typical fields include:  

\begin{itemize}
  \item \textbf{Part Designation:} free-text description (e.\,g., ``STM32F103C8 MCU 64\,KB Flash TQFP48''),
  \item \textbf{Price and MOQ:} preliminary unit price, minimum order quantity,
\end{itemize}

The challenge lies in the Part Designation column, which often merges technical specifications and the component identifier into a single free-text field. This lack of structure complicates rule-based parsing.

\section{Core Challenges}
\paragraph{C1: Inconsistent Identifier Extraction}  
A component identifier may be surrounded by package information (e.g., ``\texttt{-Tape and Reel}''), voltage values, or marketing text, making pattern-matching unreliable.  
Example:  
\begin{center}
\ttfamily PI3USB9281ZLE\_EX   20V Vbus Switch, TQFN-14, -40–85°C
\end{center}

\paragraph{C2: Incomplete Identifier-to-Parameter Mapping}
Even after successfully extracting a component identifier, there is no single public or commercial database that provides comprehensive coverage mapping part numbers to their technical parameters. Some third-party distributors offer partial mappings, but their databases are often incomplete or inconsistent, and many parts are missing or lack detailed specifications. As a result, systems cannot rely solely on these sources and must extract parameters directly from datasheets, which adds significant complexity and places greater demands on document parsing capabilities.

\paragraph{C3: Datasheet Discovery and Access}
Locating a reliable datasheet for a given component identifier typically follows this order of preference:
\begin{enumerate}
    \item \textbf{Manufacturer website:} Most trusted and up-to-date source. However, datasheets may be missing, hidden behind registration walls, or difficult to find due to poor navigation.
    \item \textbf{Distributor platforms:} Generally good coverage and easier access, but may not always have the latest or most complete datasheets.
    \item \textbf{Third-party mirror sites:} Least reliable; often present captchas, broken links, or outdated documents.
\end{enumerate}
In practice, engineers and automated systems must balance trustworthiness, accessibility, and likelihood of success, often falling back to less preferred sources when necessary to ensure datasheet retrieval.

\paragraph{C4: Parameter Extraction from Semi-Structured, Inconsistently Formatted Documents}
Datasheets are typically long, semi-structured PDFs that combine text, tables, and diagrams, but exhibit significant variation in layout and formatting across manufacturers and part types.
Technical parameters may appear in dense tables, inline prose, or embedded within figures and charts.
Standard extraction tools often struggle to correctly interpret context, units, or formatting, leading to missed or inaccurate values—especially when table structures are irregular or when key information is split between multiple sections.

\section{Formal Problem Definition and Automation Requirements}

The core problem is to automate the transformation of a noisy, semi-structured entry $q \in \mathcal{Q}$ from a Material Pricing Sheet into a validated, structured specification record:
\[
\mathbf{s} = \bigl(\mathrm{CID},\; \mathrm{Voltage_{max}},\; \mathrm{Current_{max}},\; \mathrm{TempRange}, \ldots\bigr)
\]
where each field is extracted, validated, and aligned with authoritative sources. Optional fields may be set to \texttt{null} if not present in the source.

This process must address the following, each directly mapped to a research question and technical requirement:

\begin{description}
  \item[\textbf{RQ1 / R1: Robust Component Identifier Extraction}] Accurately identify the component identifier from unstructured, noisy, and multilingual part designation fields.
  \item[\textbf{RQ2 / R2: Reliable Datasheet Retrieval}] Locate and retrieve the correct datasheet for each extracted component identifier, using a combination of distributor APIs and web search to maximise coverage and reliability.
  \item[\textbf{RQ3 / R3: Accurate Parameter Extraction}] Extract key technical parameters (e.g., voltage, current, temperature range) from datasheets using automated methods (such as vision-language models), handling diverse layouts and formats.
\end{description}

These requirements are directly aligned with the research questions and address the main technical challenges in automating the extraction of structured data from industrial procurement documents.



\section{Mapping Challenges to Research Questions}
\begin{center}
\begin{tabular}{lll}
\textbf{Challenge} & \textbf{Requirement} & \textbf{Mapped RQ} \\\hline
C1 & R1 & RQ1: Component Identifier Extraction \\
C2,C3 & R2 & RQ2: Datasheet Retrieval \\
C4 & R3 & RQ3: Parameter Extraction \\
\end{tabular}
\end{center}

This mapping clarifies how each technical challenge is addressed by a specific system requirement and evaluated through a corresponding research question in later chapters.
