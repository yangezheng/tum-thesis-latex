\chapter{Problem Setting}
\label{chapter:problem}

\section{Characteristics of Quotation Analysis Forms (QAFs)}
A Quotation Analysis Form (QAF) is a semi-structured document exchanged during early supplier negotiations.  
At BMW, QAFs appear in multiple digital formats—CSV exports, Excel sheets, and supplier-specific PDF templates.  
Typical fields include:  

\begin{itemize}
  \item \textbf{Part Designation:} free-text description (e.\,g., ``STM32F103C8 MCU 64\,KB Flash TQFP48''),
  \item \textbf{Technical Specs:} one or more numeric attributes (voltage, current, temperature limits),
  \item \textbf{Price and MOQ:} preliminary unit price, minimum order quantity,
  \item \textbf{Manufacturer Part Number (MPN):} often concatenated with package or tolerance, occasionally missing or misspelled.
\end{itemize}

The absence of a universal schema means column order, header names, and punctuation vary widely across suppliers.  
Furthermore, multilingual notations (``Gehäuse'', ``Temp. Bereich'') complicate rule-based parsing.

\section{Core Challenges}
\paragraph{C1: Inconsistent Identifier Extraction}  
An MPN may be surrounded by package codes (``\texttt{-TR}''), voltage values, or marketing text, making pattern-matching unreliable.  
Example:  
\begin{center}
\ttfamily PI3USB9281ZLE\_EX   20V Vbus Switch, TQFN-14, -40–85°C
\end{center}

\paragraph{C2: Heterogeneous Datasheet Sources}  
Even after a correct MPN is found, the corresponding datasheet might be located on:
\begin{enumerate}
  \item manufacturer domains (trusted),
  \item distributor portals (moderately trusted),
  \item third-party PDF mirrors (low trust, risk of 404 or captcha).
\end{enumerate}
Selecting the ``best'' source requires ranking by reliability, download cost, and historical success.

\paragraph{C3: Complex Document Layouts}  
Technical PDFs mix tables, figures, and footnotes.  
Numeric specs can appear in summary tables or deep in application sections.  
Standard OCR engines often split rows or mis-align columns, reducing extraction accuracy.

\paragraph{C4: Lack of Validation and Provenance}  
Classic table-extraction pipelines output values without units or context.  
For safety-critical components, engineers must know \emph{where} a value came from (page and bounding box) and whether it respects physical constraints (e.\,g., voltage $>$ 0).

\section{Formal Problem Definition}
Let $q \in \mathcal{Q}$ denote a single QAF entry containing noisy text tokens.  
The task is to compute a validated specification record  
\[
\mathbf{s} = \bigl(\mathrm{MPN},\; \mathrm{Voltage_{max}},\; \mathrm{Current_{max}},\; \mathrm{TempRange},\dots\bigr)
\]  
such that:

\begin{enumerate}
  \item \textbf{Correctness:} each numeric field equals the value printed in the authoritative datasheet.
  \item \textbf{Completeness:} mandatory fields are present; optional fields are \texttt{null} if not in the document.  
  \item \textbf{Provenance:} every field carries evidence page, bbox.  
  \item \textbf{Latency:} average end-to-end processing time $<15$ s per QAF.  
\end{enumerate}

\section{Automation Requirements}
Derived from the challenges and formal criteria, the target system must:

\begin{enumerate}
  \item \textbf{Robust MPN Extraction} – tolerate noise, multi-language descriptors, and embedded package codes.  
  \item \textbf{Adaptive Retrieval} – combine distributor APIs and web search to maximise datasheet coverage while minimising failed downloads.  
  \item \textbf{Multi-Modal Parsing} – handle tables, prose, and regulatory footnotes using a vision-language model.  
  \item \textbf{Schema Validation} – enforce type, unit, and range checks; flag anomalies.  
  \item \textbf{Traceability} – store page coordinates and raw snippets for audit.  
  \item \textbf{Scalability} – process at least 3\,000 QAF lines per day on commodity cloud hardware.  
\end{enumerate}


\section{Mapping Challenges to Research Questions}
\begin{center}
\begin{tabular}{lll}
\textbf{Challenge} & \textbf{Requirement} & \textbf{Mapped RQ} \\\hline
C1 & R1 & RQ1 \\
C2 & R2 & RQ2 \\
C3 & R3 & RQ3 \\
C4 & R4,\;R5 & RQ3 (validation aspect) \\
Performance & R6 & All RQs (latency+throughput) \\
\end{tabular}
\end{center}

% Adjust numeric labels (P1–P4, R1–R6) or add more spec fields if needed.
% Update citations once sources are chosen (e.g., OCR errors, datasheet parsing papers).

This mapping clarifies how each experimental evaluation in later chapters directly addresses an identified pain point in BMW’s procurement workflow.
