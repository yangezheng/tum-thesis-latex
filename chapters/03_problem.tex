\chapter{Problem Setting}
\label{chapter:problem}

\section{Characteristics of quotation analysis templates}
A standardised quotation analysis template is a semi-structured document exchanged during early supplier negotiations.  
In many companies, such templates appear in spreadsheet format.  
Typical fields include:  

\begin{itemize}
  \item \textbf{Part Designation:} free-text description (e.\,g., ``STM32F103C8 MCU 64\,KB Flash TQFP48''),
  \item \textbf{Price and MOQ:} preliminary unit price, minimum order quantity,
\end{itemize}

The challenge lies in the Part Designation column, which often combines technical specifications and the manufacturer part number (MPN) into a single free-text field.
This lack of structure, along with multilingual notations (e.g., Gehäuse'', Temp. Bereich''), complicates rule-based parsing.

\section{Core Challenges}
\paragraph{C1: Inconsistent Identifier Extraction}  
An MPN may be surrounded by package codes (``\texttt{-TR}''), voltage values, or marketing text, making pattern-matching unreliable.  
Example:  
\begin{center}
\ttfamily PI3USB9281ZLE\_EX   20V Vbus Switch, TQFN-14, -40–85°C
\end{center}

\paragraph{C2: Incomplete Identifier-to-Parameter Mapping}
Even after successfully extracting an MPN, there is no single public or commercial database that provides comprehensive coverage mapping part numbers to their technical parameters. Some third-party distributors offer partial mappings, but their databases are often incomplete or inconsistent, and many parts are missing or lack detailed specifications. As a result, systems cannot rely solely on these sources and must extract parameters directly from datasheets, which adds significant complexity and places greater demands on document parsing capabilities.

\paragraph{C3: Datasheet Discovery and Access}
Locating a reliable datasheet for a given MPN typically follows this order of preference:
\begin{enumerate}
    \item \textbf{Manufacturer website:} Most trusted and up-to-date source. However, datasheets may be missing, hidden behind registration walls, or difficult to find due to poor navigation.
    \item \textbf{Distributor platforms:} Generally good coverage and easier access, but may not always have the latest or most complete datasheets.
    \item \textbf{Third-party mirror sites:} Least reliable; often present captchas, broken links, or outdated documents.
\end{enumerate}
In practice, engineers and automated systems must balance trustworthiness, accessibility, and likelihood of success, often falling back to less preferred sources when necessary to ensure datasheet retrieval.

\paragraph{C4: Parameter Extraction from Semi-Structured, Inconsistently Formatted Documents}
Datasheets are typically long, semi-structured PDFs that combine text, tables, and diagrams, but exhibit significant variation in layout and formatting across manufacturers and part types.
Technical parameters may appear in dense tables, inline prose, or embedded within figures and charts.
Standard extraction tools often struggle to correctly interpret context, units, or formatting, leading to missed or inaccurate values—especially when table structures are irregular or when key information is split between multiple sections.

\paragraph{C5: Parameter Alignment Across Sources}
Technical parameters for a given part are often available from multiple sources, such as distributor APIs and manufacturer datasheets. However, these sources may provide overlapping, incomplete, or even conflicting information—differing in terminology, units, or value precision. In practice, our pipeline first attempts to retrieve parameters from distributor APIs, and only extracts missing fields from datasheets as needed. This approach introduces the challenge of aligning and reconciling parameter values across sources. Even with this fallback strategy, discrepancies can arise (e.g., different units, naming conventions, or value ranges). Addressing this requires at minimum a rule-based alignment system: normalizing units, deduplicating fields, and defining source priorities or conflict-resolution rules. More advanced solutions—such as confidence scoring or manual review of conflicts—remain open research questions and are identified as areas for future work.

\paragraph{C6: Provenance and Traceability Requirements}
For every extracted parameter, the system must record detailed provenance information to ensure full traceability. This includes documenting whether the value was obtained from a distributor API or extracted from a datasheet. If the source is a datasheet, the provenance must specify the exact URL from which the datasheet was downloaded and the page number where the parameter was found. Additional metadata such as the document name, table or bounding box location, and source type should also be captured. Comprehensive provenance enables reliable auditing, validation, and engineer trust in the extracted data.

\section{Formal Problem Definition}
Let $q \in \mathcal{Q}$ represent a single entry from a quotation analysis template, typically a noisy, semi-structured text string.  
The objective is to transform $q$ into a validated, structured specification record:
\[
\mathbf{s} = \bigl(\mathrm{MPN},\; \mathrm{Voltage_{max}},\; \mathrm{Current_{max}},\; \mathrm{TempRange}, \ldots\bigr)
\]
such that the following criteria—each corresponding to a research question—are satisfied:

\begin{enumerate}
  \item \textbf{MPN Extraction (RQ1):} The Manufacturer Part Number is accurately identified from the noisy input.
  \item \textbf{Datasheet Retrieval (RQ2):} The correct datasheet is reliably located and retrieved for the extracted MPN.
  \item \textbf{Parameter Extraction (RQ3):} Key technical parameters (e.g., voltage, current, temperature range) are precisely extracted from the datasheet, matching the authoritative source.
  \item \textbf{Provenance (RQ4):} For every extracted field, detailed provenance is recorded, including the data source (API or datasheet), document name or URL, and page number or location.
\end{enumerate}

Optional fields may be set to \texttt{null} if not present in the source.

\section{Automation Requirements}
To address the above problem and directly map to the research questions, the system must:

\begin{enumerate}
  \item \textbf{R1: Robust MPN Extraction (RQ1)} – Accurately extract Manufacturer Part Numbers from unstructured, noisy, and multilingual part designation fields.
  \item \textbf{R2: Reliable Datasheet Retrieval (RQ2)} – Retrieve the correct datasheet for each MPN using a combination of distributor APIs and web search, maximising coverage and reliability.
  \item \textbf{R3: Accurate Parameter Extraction (RQ3)} – Extract key technical parameters from datasheets using automated methods (e.g., vision-language models), handling diverse layouts and formats.
  \item \textbf{R4: End-to-End Provenance (RQ4)} – For every extracted parameter, record detailed provenance information, including source type, document URL or identifier, and page or location, to ensure full traceability.
\end{enumerate}

These requirements are directly aligned with the research questions and address the main technical challenges in automating the extraction of structured data from industrial procurement documents.


% mapping
% \begin{description}
%   \item[RQ1.] How accurately can a language-model agent extract correct MPNs from semi-structured standardised quotation analysis template strings?
%   \item[RQ2.] Can cooperative retrieval agents-one using a distributor API, the other web search—achieve near-complete datasheet coverage across diverse suppliers and part formats??
%   \item[RQ3.] What precision and recall can a vision-language parsing agent attain on key electrical specifications?
%   \item[RQ5.] How can the system be designed to ensure traceability and provenance of extracted data?
% \end{description}


\section{Mapping Challenges to Research Questions}
\begin{center}
\begin{tabular}{lll}
\textbf{Challenge} & \textbf{Requirement} & \textbf{Mapped RQ} \\\hline
C1 & R1 & RQ1: MPN Extraction \\
C4 & R3 & RQ3: Parameter Extraction \\
C5 & R2, R3 & RQ2, RQ3: Datasheet Retrieval, Parameter Alignment \\
C6 & R4 & RQ4: Provenance and Traceability \\
\textit{(Cross-cutting)} & All & All RQs (system-level integration) \\
\end{tabular}
\end{center}

This mapping clarifies how each technical challenge is addressed by a specific system requirement and evaluated through a corresponding research question in later chapters.
